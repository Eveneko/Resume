% LaTeX source of my resume
% =========================

% Commented for easy reuse... ;)

% See the `README.md` file for more info.

% This file is licensed under the CC-NC-ND Creative Commons license.


% Start a document with the here given default font size and paper size.
\documentclass[10pt,a4paper]{article}

% Set the page margins.
\usepackage[a4paper,margin=0.75in]{geometry}

% Setup the language.
\usepackage[english]{babel}
\hyphenation{Some-long-word}

% Makes resume-specific commands available.
\usepackage{resume}




\begin{document}  % begin the content of the document
\sloppy  % this to relax whitespacing in favour of straight margins


% title on top of the document
\maintitle{胡玉斌}{2, 3, 1999}{上次修改在 \today}
\nobreakvspace{0.3em}  % add some page break averse vertical spacing

% \noindent prevents paragraph's first lines from indenting
% \mbox is used to obfuscate the email address
% \sbull is a spaced bullet
% \href well..
% \\ breaks the line into a new paragraph

\begin{multicols}{2}
\noindent 本科生
\\
计算机科学与工程系,
\\
\href{https://www.sustech.edu.cn/}{南方科技大学}
\\
\\
中国广东省深圳市
\\
南山区,学苑大道1088号
\\
南方科技大学,创园10栋
\\
个人主页:\href{https://eveneko.com}{eveneko.com}
\\
邮箱:\noindent\href{mailto:huyubin0203@gmail.com}{huyubin0203\mbox{}@\mbox{}gmail.com}
\\
电话:\textsmaller{+}(86)18823344853
\\
Github:\href{https://github.com/Eveneko}{github.com/Eveneko}
\\
LinkedIn:\href{https://www.linkedin.com/in/yubin-hu-7855861a0/}{linkedin/yubin-hu}
\\
Codeforces:\href{http://codeforces.com/profile/SUSTech-Neko}{codeforces/SUSTech-Neko}
\end{multicols}

% \spacedhrule{0.9em}{-0.4em}  % a horizontal line with some vertical spacing before and after

% \roottitle{Summary}  % a root section title

% \vspace{-1.3em}  % some vertical spacing
% \begin{multicols}{2}  % open a multicolumn environment
% \noindent \emph{Entrepeneurial geek with roots in the open source movement. Passionately enabling software-related teams to deliver. Creates/\,structures/\,improves processes. Architects and helps to implement future-proof solutions. Coaches both individuals and groups.}
% \\
% \\
% At the age of seven (1989) Cies wrote his first lines of code in a \acr{LOGO}-like language on an \acr{MSX} (pre-\acr{PC}).  Two years later he attended a conference on an emerging new technology, the Internet, at the Erasmus University from which he would graduate 16 years later (2000) with a degree in Business Computer Science.

% After being introduced to the open source movement in 1997, he taught himself a variety of skills including system administration and programming (Bash, Python, Ruby \& \CPP).  By 2002 he got his pet project \acr{KT}urtle ---a zero-entry-barrier programming environment--- included in \acr{KDE}'s \emph{edu} module, and thereby in almost every Linux distribution.

% Extensively travelled Europe and Asia after graduating. On return his professional life became a mix of hands-on work in startups and web software agencies, while settling as a husband and father in personal life.
% \end{multicols}


\spacedhrule{0em}{-0.4em}

\roottitle{教育经历}

\headedsection
  {\href{https://www.sustech.edu.cn/}{南方科技大学}}
  {\textsc{中国广东省深圳市}} {%
  \edusubsection
    {计算机科学与技术工学学士}
    {2017 -- 至今}
    {学术导师: \href{https://yepangliu.github.io}{刘烨庞}}
    {GPA: 3.52/4.0}
}


\spacedhrule{0.5em}{-0.4em}

\roottitle{工作经历}

\headedsection
  {数据结构与算法分析}
  {\textsc{南方科技大学}} {%
  \headedsubsection
    {助教}
    {2019.9 -- 2019.12}
    {}
}

\headedsection
  {算法分析与设计}
  {\textsc{南方科技大学}} {%
  \headedsubsection
    {助教}
    {2020.2 -- 至今}
    {}
}

% \vspace{-0.2em}
% \begin{center}
%   \emph{\small 可在 \href{http://www.linkedin.com/in/yubin-hu-7855861a0/}{Yubin Hu' LinkedIn profile} 上查看完整经历}
% \end{center}


\spacedhrule{0.5em}{-0.4em}

\roottitle{技能}

\begin{tabular}{@{\bfseries} l @{\hspace{6ex}} l}
\hspace{\newparindent}已修课程:& 人工智能,计算机网络,计算机组成原理,数据结构与算法分析,面向对象分析等
\\
\hspace{\newparindent}编程语言:& C, C++, HTML, Java, JavaScript, Kotlin, Python, SQL
\\
\hspace{\newparindent}Web应用框架:& Django, Flask
\end{tabular}

\spacedhrule{1.6em}{-0.4em}


\roottitle{课题研究项目}

\headedsection
  {WebAssembly Fuzzing Test \faGithub~\href{https://github.com/Eveneko/WASM-Fuzzing-Platform}{Code}}
  {\textsc{南方科技大学}} {%
  \headedsubsection
    {学术导师: 刘烨庞}
    {2019.9 - 2019.12}
    {
      \begin{itemize} 
        \item 在WebAssembly的抽象语法树层面上进行变异并且在WebAssembly虚拟机上进行大规模模糊测试来寻找WebAssembly解释器的错误。
        \item 动态地对一些测试输入执行程序,并使用原始版本的等效变量检查输出是否一致。
        \item 构建了WASM Fuzzing平台,以方便进行测试,并更好地展示研究工作。
      \end{itemize}
    }
}

\spacedhrule{0.5em}{-0.4em}

\roottitle{课程项目}

\headedsection
  {Online Algorithm Store \faGithub~\href{https://github.com/Eveneko/Astor}{Code}~\faLink~\href{https://astor.eveneko.com}{Demo}} 
  {\textsc{南方科技大学}} {%
  \headedsubsection
    {主要开发者}
    {2019.9 - 2020.1}
    {
      \\
      Astor是一个在线算法商店和服务集群的结合. 它是为相关物流工作者提供在线算法购买以及算法部署执行的系统。
      \begin{itemize}
        \item 实现了算法的购买,算法的部署,docker上算法的执行,和物流微服务API的通信以及部署到云服务器上。
        \item 执行的算法包括图论算法和调度算法。
      \end{itemize}
    }
}

\headedsection
  {AirGesture \faGithub~\href{https://github.com/Eveneko/AirGesture}{Code}}
  {\textsc{南方科技大学}} {%
  \headedsubsection
    {主要开发者}
    {2019.10 - 2019.12}
    {
      \\
      AirGesture是一个利用识别双手特定手势来操控计算机行为的系统。
      \begin{itemize}
        \item 设计了一个样例,通过识别手势来控制Google小恐龙游戏。
        \item 使用华为云平台以及华为AI设备\textbf{Hilens Kits},使用Flask进行设备通信等功能。
      \end{itemize}
    }
}

\headedsection
  {Influence Maximization \faGithub~\href{https://github.com/Eveneko/SUSTech-Courses/tree/master/CS303-Artifical-Intelligence/IMP}{Code}}
  {\textsc{南方科技大学}} {%
  \headedsubsection
    {个人项目}
    {2019.10 - 2019.11}
    {
      \\
      给定一个具有作为边缘权重和扩散概率的社交网络,应选择哪个k个节点进行信息的初始注入,以最大程度地影响网络。
      \begin{itemize}
        \item 并行执行IMP算法,加入Lazy标记优化时间。
        \item 在性能测试中得到了很高的分数。 (98/100)
      \end{itemize}
    }
}

\headedsection
  {Tetris \faGithub~\href{https://github.com/Eveneko/Tetris}{Code}}
  {\textsc{南方科技大学}} {%
  \headedsubsection
    {主要开发者}
    {2019.10 - 2019.12}
    {
      \\
      俄罗斯方块是一款经典游戏,实现了在STM32F03RC上运行并添加了一些可以使游戏变得更加有趣更加困难的功能。
      \begin{itemize}
        \item 实现了可制定的方块。
        \item 实现了分级难度。
      \end{itemize}
    }
}


\spacedhrule{0.5em}{-0.4em}

\roottitle{荣誉和奖项}

\headedsection
  {广东省程序设计竞赛}
  {\textsc{中国,广东省}} {%
  \headedsubsection
    {\textit {三等奖}}
    {2019.5}
    {}
}

\headedsection
  {美国大学生数学建模竞赛}
  {\textsc{美国数学及其应用联合会}} {%
  \headedsubsection
    {\textit {S奖}}
    {2019.3}
    {}
}

\headedsection
  {年度优秀学生}
  {\textsc{南方科技大学}} {%
  \headedsubsection
    {\textit {三等奖}}
    {2018.11}
    {}
}


\spacedhrule{0.5em}{-0.4em}

\roottitle{课外活动}

\headedsection
  {学术新闻社部长}
  {\textsc{学生新闻社, 南方科技大学}} {%
  \headedsubsection
    {新媒体部}
    {2018.5 -- 2019.1}
    {}
}

\spacedhrule{0.5em}{-0.4em}

\roottitle{语言能力}

\inlineheadsection % special section that has an inline header with a 'hanging' paragraph
  {中文}
  {\emph{(母语)}}

  \headedsection
  {英语}
  {} {%
  \headedsubsection
    {全国大学英语六级考试 \emph{491}}
    {2019.6}
    {}
}

\end{document}
