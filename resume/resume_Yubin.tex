% LaTeX source of my resume
% =========================

% Commented for easy reuse... ;)

% See the `README.md` file for more info.

% This file is licensed under the CC-NC-ND Creative Commons license.


% Start a document with the here given default font size and paper size.
\documentclass[10pt,a4paper]{article}

% Set the page margins.
\usepackage[a4paper,margin=0.75in]{geometry}

% Setup the language.
\usepackage[english]{babel}
\hyphenation{Some-long-word}

% Makes resume-specific commands available.
\usepackage{resume}




\begin{document}  % begin the content of the document
\sloppy  % this to relax whitespacing in favour of straight margins


% title on top of the document
\maintitle{Yubin Hu(胡玉斌)}{February 3, 1999}{Last update on \today}
\nobreakvspace{0.3em}  % add some page break averse vertical spacing

% \noindent prevents paragraph's first lines from indenting
% \mbox is used to obfuscate the email address
% \sbull is a spaced bullet
% \href well..
% \\ breaks the line into a new paragraph

\begin{multicols}{2}
\noindent UG Student
\\
Department of Computer Science and Engineering,
\\
\href{https://www.sustech.edu.cn/}{Southern University of Science and Technology}
\\
\\
Block 10, Innovation Park, SUSTech
\\
1088 Xueyuan Blvd., Nanshan District
\\
Shenzhen, Guangdong, China
\\
Home Page: \href{https://eveneko.com}{eveneko.com}
\\
Email: \noindent\href{mailto:huyubin0203@gmail.com}{huyubin0203\mbox{}@\mbox{}gmail.com}
\\
Phone: \textsmaller{+}(86)18823344853
\\
Github: \href{https://github.com/Eveneko}{github.com/Eveneko}
\\
Linkedin: \href{https://www.linkedin.com/in/yubin-hu-7855861a0/}{linkedin/yubin-hu}
\\
Codeforces: \href{http://codeforces.com/profile/SUSTech-Neko}{codeforces/SUSTech-Neko}
\end{multicols}

% \spacedhrule{0.9em}{-0.4em}  % a horizontal line with some vertical spacing before and after

% \roottitle{Summary}  % a root section title

% \vspace{-1.3em}  % some vertical spacing
% \begin{multicols}{2}  % open a multicolumn environment
% \noindent \emph{Entrepeneurial geek with roots in the open source movement. Passionately enabling software-related teams to deliver. Creates/\,structures/\,improves processes. Architects and helps to implement future-proof solutions. Coaches both individuals and groups.}
% \\
% \\
% At the age of seven (1989) Cies wrote his first lines of code in a \acr{LOGO}-like language on an \acr{MSX} (pre-\acr{PC}).  Two years later he attended a conference on an emerging new technology, the Internet, at the Erasmus University from which he would graduate 16 years later (2000) with a degree in Business Computer Science.

% After being introduced to the open source movement in 1997, he taught himself a variety of skills including system administration and programming (Bash, Python, Ruby \& \CPP).  By 2002 he got his pet project \acr{KT}urtle ---a zero-entry-barrier programming environment--- included in \acr{KDE}'s \emph{edu} module, and thereby in almost every Linux distribution.

% Extensively travelled Europe and Asia after graduating. On return his professional life became a mix of hands-on work in startups and web software agencies, while settling as a husband and father in personal life.
% \end{multicols}


\spacedhrule{0em}{-0.4em}

\roottitle{Education}

\headedsection
  {\href{https://www.sustech.edu.cn/}{Southern University of Science and Technology}}
  {\textsc{Shenzhen, Guangdong, China}} {%
  \edusubsection
    {BEng in Computer Science and Technology \textnormal{~(\acr{CS})}}
    {2017 -- Present}
    {Supervisor: \href{https://yepangliu.github.io}{Yepang Liu}}
    {GPA: 3.52/4.0}
}


\spacedhrule{0.5em}{-0.4em}

\roottitle{Experience}

\headedsection
  {Data Structure and Algorithm Analysis}
  {\textsc{SUSTech}} {%
  \headedsubsection
    {Teaching Assistant}
    {Sep.2019 -- Dec.2019}
    {}
}

\headedsection
  {Algorithms Design and Analysis}
  {\textsc{SUSTech}} {%
  \headedsubsection
    {Teaching Assistant}
    {Feb.2020 -- Present}
    {}
}

% \vspace{-0.2em}
% \begin{center}
%   \emph{\small Please refer to \href{http://www.linkedin.com/in/yubin-hu-7855861a0/}{Yubin Hu' LinkedIn profile} for a more complete list of work experience along with recommendations.}
% \end{center}


\spacedhrule{0.5em}{-0.4em}

\roottitle{Skills}

\begin{tabular}{@{\bfseries} l @{\hspace{6ex}} l}
\hspace{\newparindent}Selected Core Courses: & Artificial Intelligence, Computer Networks, Computer 
\\
 & Organization, Data Structure and Algorithm Analysis, 
\\
 & Object-Oriented Programming, etc.
\\
\hspace{\newparindent}Programming Language: & C, C++, HTML, Java, JavaScript, Kotlin, Python, SQL
\\
\hspace{\newparindent}Web application framework: & Django, Flask
\end{tabular}

\spacedhrule{1.6em}{-0.4em}


\roottitle{Research Project}

\headedsection
  {WebAssembly Fuzzing Test \faGithub~\href{https://github.com/Eveneko/WASM-Fuzzing-Platform}{code}}
  {\textsc{SUSTech}} {%
  \headedsubsection
    {Supervisor: Yepang Liu}
    {Sep.2019 - Dec.2019}
    {
      \begin{itemize} 
        \item Made some \textbf{mutation} on WebAssembly on \textbf{abstract syntax tree level} and \\
        used some large scale test data to do a fuzzing test on the \textbf{Web-Assembly \\
        virtual machines} to find the bugs in WebAssembly compilers.
        \item Executed a program on some test inputs dynamically and used equivalent \\
        variants of the original to check whether the outputs were coherent.
        \item Built a \textbf{WASM Fuzzing Platform} for convenient testing and also a better \\
        visible demonstration of research work.
      \end{itemize}
    }
}


\spacedhrule{0.5em}{-0.4em}

\roottitle{Notable Course Project}

\headedsection
  {Online Algorithm Store \faGithub~\href{https://github.com/Eveneko/Astor}{Code}~\faLink~\href{https://astor.eveneko.com}{Demo}} 
  {\textsc{SUSTech}} {%
  \headedsubsection
    {Main developer}
    {Sep.2019 - Jan.2020}
    {
      \\
      Astor is a \textbf{combination of an online algorithm store and a service cluster}. Its \\
      purpose is to provide logistics staff to buy algorithms online and run examples.
      \begin{itemize}
        \item Implemented\textbf{algorithm purchase}, \textbf{algorithm deployment}, \textbf{algorithm \\
        execution} by docker, \textbf{connects a logistics microservice API}, and \textbf{deploys} \\
        it to a cloud server.
      \end{itemize}
    }
}

\headedsection
  {AirGesture \faGithub~\href{https://github.com/Eveneko/AirGesture}{Code}}
  {\textsc{SUSTech}} {%
  \headedsubsection
    {Main developer}
    {Oct.2019 - Dec.2019}
    {
      \\
      AirGesture is an \textbf{AI recognition system} that controls computer behavior by \\
      recognizing \textbf{specific gestures} of both hands.
      \begin{itemize}
        \item Designed gestures and made a demo that controls \textbf{Google dinosaur game}.
        \item Utilized \textbf{ModelArts} from Huawei Cloud Platform, AI device \textbf{Hilens Kits} \\
        and \textbf{Flask} for getting the messages for device.
      \end{itemize}
    }
}

\headedsection
  {Influence Maximization \faGithub~\href{https://github.com/Eveneko/SUSTech-Courses/tree/master/CS303-Artifical-Intelligence/IMP}{Code}}
  {\textsc{SUSTech}} {%
  \headedsubsection
    {individual project}
    {Oct.2019 - Nov.2019}
    {
      \\
      Given a social network with diffusion probabilities as edge weights and an integer \\
      k, which k nodes should be chosen for initial injection of information to maximize \\
      influence in the network.
      \begin{itemize}
        \item Executed the algorithm in parallel and use Lazy markers to greatly optimize time.
        \item Got high score in the performance contest in class. (98/100)
      \end{itemize}
    }
}

\headedsection
  {Tetris \faGithub~\href{https://github.com/Eveneko/Tetris}{Code}}
  {\textsc{SUSTech}} {%
  \headedsubsection
    {Main developer}
    {Oct.2019 - Dec.2019}
    {
      \\
      Tetris is a classic game. Implemented it on STM32F103RC and added some of \\
      own ideas to make it more interesting and difficult.
      \begin{itemize}
        \item Implemented \textbf{customizable} bricks.
        \item Achieved grading difficulty.
      \end{itemize}
    }
}


\spacedhrule{0.5em}{-0.4em}

\roottitle{Honors and Awards}

\headedsection
  {Guangdong Collegiate Programming Contest}
  {\textsc{Guangdong Province, China}} {%
  \headedsubsection
    {\textit {Third Prize}}
    {May.2019}
    {}
}

\headedsection
  {Interdisciplinary Contest in Modeling}
  {\textsc{the Consortium for Mathematics and Its Application}} {%
  \headedsubsection
    {\textit {Successful Participant}}
    {Apr.2019}
    {}
}

\headedsection
  {Annual Outstanding Student}
  {\textsc{SUSTech}} {%
  \headedsubsection
    {\textit {Third Prize}}
    {Nov.2018}
    {}
}


\spacedhrule{0.5em}{-0.4em}

\roottitle{Extra-Curricular}

\headedsection
  {Team Leader in Student News Agency}
  {\textsc{Team Leader in Student News Agency, SUSTech}} {%
  \headedsubsection
    {New Media Department}
    {May.2018 -- Jan.2019}
    {}
}

\spacedhrule{0.5em}{-0.4em}

\roottitle{Language}

\inlineheadsection % special section that has an inline header with a 'hanging' paragraph
  {Chinese}
  {\emph{(mother tongue)}}

  \headedsection
  {English}
  {} {%
  \headedsubsection
    {CET6 \emph{491}}
    {Jun.2019}
    {}
}

\end{document}
