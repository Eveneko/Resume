% LaTeX source of my resume
% =========================

% Commented for easy reuse... ;)

% See the `README.md` file for more info.

% This file is licensed under the CC-NC-ND Creative Commons license.


% Start a document with the here given default font size and paper size.
\documentclass[10pt,a4paper]{article}

% Set the page margins.
\usepackage[a4paper,margin=0.75in]{geometry}

% Setup the language.
\usepackage[english]{babel}
\hyphenation{Some-long-word}

% Makes resume-specific commands available.
\usepackage{resume}




\begin{document}  % begin the content of the document
\sloppy  % this to relax whitespacing in favour of straight margins


% title on top of the document
\maintitle{胡玉斌}{2, 3, 1999}{上次修改在 \today}
\nobreakvspace{0.3em}  % add some page break averse vertical spacing

% \noindent prevents paragraph's first lines from indenting
% \mbox is used to obfuscate the email address
% \sbull is a spaced bullet
% \href well..
% \\ breaks the line into a new paragraph

\begin{multicols}{2}
\noindent 电话:\textsmaller{+}(86)18823344853
\\
个人主页:\href{https://eveneko.com}{eveneko.com}
\\
邮箱:\noindent\href{mailto:huyubin0203@gmail.com}{huyubin0203\mbox{}@\mbox{}gmail.com}
\\
Github:\href{https://github.com/Eveneko}{github.com/Eveneko}
\\
LinkedIn:\href{https://www.linkedin.com/in/yubin-hu-7855861a0/}{linkedin/yubin-hu}
\\
Codeforces:\href{http://codeforces.com/profile/SUSTech-Neko}{codeforces/SUSTech-Neko}
\end{multicols}

% \spacedhrule{0.9em}{-0.4em}  % a horizontal line with some vertical spacing before and after

% \roottitle{Summary}  % a root section title

% \vspace{-1.3em}  % some vertical spacing
% \begin{multicols}{2}  % open a multicolumn environment
% \noindent \emph{Entrepeneurial geek with roots in the open source movement. Passionately enabling software-related teams to deliver. Creates/\,structures/\,improves processes. Architects and helps to implement future-proof solutions. Coaches both individuals and groups.}
% \\
% \\
% At the age of seven (1989) Cies wrote his first lines of code in a \acr{LOGO}-like language on an \acr{MSX} (pre-\acr{PC}).  Two years later he attended a conference on an emerging new technology, the Internet, at the Erasmus University from which he would graduate 16 years later (2000) with a degree in Business Computer Science.

% After being introduced to the open source movement in 1997, he taught himself a variety of skills including system administration and programming (Bash, Python, Ruby \& \CPP).  By 2002 he got his pet project \acr{KT}urtle ---a zero-entry-barrier programming environment--- included in \acr{KDE}'s \emph{edu} module, and thereby in almost every Linux distribution.

% Extensively travelled Europe and Asia after graduating. On return his professional life became a mix of hands-on work in startups and web software agencies, while settling as a husband and father in personal life.
% \end{multicols}


\spacedhrule{0em}{-0.4em}

\roottitle{教育经历}

\nopagebreak[4]
  \begin{indentsection}
    \item[]
    \href{https://www.sustech.edu.cn/}{\textbf{南方科技大学(SUSTech)}}
    \hfill{2017年9月 -- 2021年6月}
    \item[]
    学术导师: \href{https://yepangliu.github.io}{\textit{刘烨庞}}
    \hfill{预期毕业}
    \item[]
    \textit{计算机科学与技术工学学士}
    \hfill{{GPA: 3.52/4.0}}
  \end{indentsection}
\nopagebreak[4]

\spacedhrule{0.5em}{-0.4em}

\roottitle{实习经历}

\nopagebreak[4]
  \begin{indentsection}
    \item[]
    腾讯科技有限公司,
    中国,深圳
    \hfill{2020年6月-至今}
  \end{indentsection}
\nopagebreak[4]

\spacedhrule{0.5em}{-0.4em}

\roottitle{荣誉和奖项}

\nopagebreak[4]
  \begin{indentsection}
    \item[]
    广东省程序设计竞赛,
    \textit{三等奖}
    ,中国,广东省
    \hfill{2019年5月}
  \end{indentsection}
\nopagebreak[4]

\nopagebreak[4]
  \begin{indentsection}
    \item[]
    美国大学生数学建模竞赛,
    \textit{S奖}
    ,美国数学及其应用联合会
    \hfill{2019年3月}
  \end{indentsection}
\nopagebreak[4]

\spacedhrule{0.5em}{-0.4em}

\roottitle{专业课程}

\bodytext {数据结构与算法分析,算法设计与分析,人工智能}
\bodytext {计算机组成原理,嵌入式系统与微机原理}
\bodytext {面向对象分析与设计,软件工程}
\bodytext {数据库原理,计算机网络,操作系统}


\spacedhrule{1.6em}{-0.4em}


\roottitle{课题研究项目}

\headedsection
  {\textbf{WebAssembly Fuzzing Test} \faGithub~\href{https://github.com/Eveneko/WASM-Fuzzing-Platform}{Code}}
  {\textsc{2019年9月 - 2019年12月}} {%
  \prosubsection
    {学术导师: 刘烨庞}
    {南方科技大学}
    {
      \textbf{项目描述:}
      在WebAssembly的抽象语法树层面上进行变异并且在WebAssembly虚拟机上进行大规模模糊测试来寻找WebAssembly解释器的错误。动态地对一些测试输入执行程序,并使用原始版本的等效变量检查输出是否一致。构建了WASM Fuzzing平台,以方便进行测试,并更好地展示研究工作。
    }
    {
      \textbf{实现技术:}
      Flask框架;Fuzzing Test。
    }
}

\headedsection
  {\textbf{Understanding Real-World Performance Bugs in Go}}
  {\textsc{2019年3月 - 2020年6月}} {%
  \prosubsection
    {学术导师: 刘烨庞}
    {南方科技大学}
    {
      \textbf{项目描述:}
      对19个大型开源项目对提交历史进行分析,提取其中对性能bug,并对其分类,探索它们对特性,区别以及如何被修复,为go项目开发提供良好指导。
    }
    {
      \textbf{实现技术:}
      Go,Pprogf
    }
}

\spacedhrule{0.5em}{-0.4em}

\roottitle{课程项目}

\headedsection
  {\textbf{Online Algorithm Store} \faGithub~\href{https://github.com/Eveneko/Astor}{Code}~\faLink~\href{https://astor.eveneko.com}{Demo}} 
  {\textsc{2019年9月 - 2020年1月}} {%
  \prosubsection
    {主要开发者}
    {南方科技大学}
    {
      \textbf{项目描述:}
      Astor是一个在线算法商店和服务集群的结合. 它是为相关物流工作者提供在线算法购买以及算法部署执行的系统。实现了算法的购买,算法的部署,docker上算法的执行,和物流微服务API的通信以及部署到云服务器上。执行的算法包括图论算法和调度算法。
    }
    {
      \textbf{实现技术:}
      Django框架;Bootstrap;MySQL;Docker。
    }
}

\end{document}
