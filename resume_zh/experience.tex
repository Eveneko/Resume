%-------------------------------------------------------------------------------
%	SECTION TITLE
%-------------------------------------------------------------------------------
\cvsection{Experience}


%-------------------------------------------------------------------------------
%	CONTENT
%-------------------------------------------------------------------------------
\begin{cventries}

%---------------------------------------------------------
\cventry
{车端软件开发工程师} % Job title
{蔚来, 端侧工程部, 自动驾驶研发部门} % Organization
{北京, 中国} % Location
{Feb. 2023 - 至今} % Date(s)
{
  \begin{cvitems} % Description(s) of tasks/responsibilities
    \item {技术栈: C++, Python, Kernel;}
    \item {参与了车端应用的开发工作, 针对车端服务出现的CPU资源使用超标、可能的内存泄漏问题, 以及实时性服务未能得到CPU资源保障的情况, 利用cgroup技术对车端服务进行系统资源的监控和限制, 以确保系统的稳定运行;}
    \item {针对自动驾驶场景中的连续突发资源需求问题, 验证并部署Burstable CFS带宽控制方案, 允许CPU突发使用, 从而有效降低服务时延, 阶段性地解决问题。目前, 为进一步提高系统性能和稳定性, 正在研发自适应的CFS带宽控制方案以满足自动驾驶的CPU需求;}
    \item {在仿真环境和实车中, 复现已知的性能相关问题, 运用perf、trace等工具, 有效地收集并分析车端系统性能消耗。通过数据可视化, 进一步分析了系统资源使用情况及可能存在的优化空间和潜在漏洞;}
  \end{cvitems}
}

%---------------------------------------------------------
  \cventry
    {事务型开发实习生} % Job title
    {腾讯, CSIG云与智慧产业事业群, 乘车码产品中心} % Organization
    {深圳, 中国} % Location
    {Jun. 2020 - Aug. 2020} % Date(s)
    {
      \begin{cvitems} % Description(s) of tasks/responsibilities
        \item {技术栈: C++, MySQL, Redis, Nginx, Kafka;}
        \item {参与微信乘车码日常需求开发, 测试和上线部署, 例如实现地铁人流密度监控可视化和乘车优惠发放功能;}
        \item {搭建配置文件平台, 更有效地管理各种环境的配置, 确保配置文件的一致性和准确性, 并降低配置错误造成的风险;}
      \end{cvitems}
    }

%---------------------------------------------------------
  \cventry
    {贡献者} % Job title
    {Github开源项目贡献} % Organization
    {} % Location
    {Mar. 2020 - Jun. 2020} % Date(s)
    {
      \begin{cvitems} % Description(s) of tasks/responsibilities
        \item {\href{https://github.com/assertj/assertj}{AssertJ}, 2.3k star, 9 issues, 一个用于编写更富表达力和可读性的断言的Java开源库;}
        \item {修复已知issue漏洞并新增单元测试, 优化断言性能, 增加断言失败堆栈信息;}
      \end{cvitems}
    }
%---------------------------------------------------------
\end{cventries}
