%-------------------------------------------------------------------------------
%	SECTION TITLE
%-------------------------------------------------------------------------------
\cvsection{Researchs}
%-------------------------------------------------------------------------------
%	CONTENT
%-------------------------------------------------------------------------------
\begin{cventries}
%---------------------------------------------------------
\cventry
    {Researchers, 导师: \href{https://howiepku.github.io/index.html}{王浩宇}} % Role 
    {\href{https://dl.acm.org/doi/10.1145/3597926.3598064}{Eunomia: Enabling User-Specified Fine-Grained Search in Symbolically Executing WebAssembly Binaries, ISSTA '23}} % Projects Name 
    {北京, 中国} % Location
    {Jan. 2022 - Nov. 2022} % Date(s)
    {
	\begin{cvitems} % Description(s) of tasks/responsibilities
        \item {参与开发实现首个支持WebAssembly全部特性的符号执行引擎, 同时支持分析多种语言编写的应用程序;} 
        \item {引入了领域特定语言(DSL)来定义局部搜索策略环境符号执行的路径爆炸问题, 支持具有局部领域知识的细粒度搜索;}
        \item {允许用户为程序的不同部分指定局部搜索策略, 并为不同局部搜索策略隔离变量的上下文以避免冲突;}
        \item {成功验证了六个已知缺陷, 并且发现了Collections-C中的两个新的0-day漏洞;}
      \end{cvitems}
    }
%---------------------------------------------------------

\end{cventries}
